\addcontentsline{toc}{section}{Wstęp} % Dodaj "WSTĘP" do spisu treści
%-------------------------------------------------------------
%-------------------------------------------------------------
%-------------------------------------------------------------
%                          WSTĘP
%-------------------------------------------------------------
%-------------------------------------------------------------
%-------------------------------------------------------------

\setstretch{1.5}
\setlength{\parindent}{1.25cm} % Ustaw wcięcie akapitu

\section*{Wstęp} % * aby nie dodało się do spisu treści ale przyjęło format nagłówkowy
\begin{spacing}{1.5} % Zaczynamy interlinię 1.5

    W erze cyfrowej technologie audiowizualne doświadczają nieustannego rozwoju, wnosząc znaczący wkład w różne dziedziny – od produkcji muzycznej po inżynierię dźwięku. W szczególności, branża muzyczna przeżywa rewolucję dzięki postępowi w dziedzinie oprogramowania do przetwarzania i produkcji dźwięku. Wtyczki VST (Virtual Studio Technology), będące w centrum tej rewolucji, umożliwiają twórcom muzyki zastosowanie zaawansowanych technik przetwarzania dźwięku w środowisku cyfrowym. Niniejsza praca inżynierska skupia się na jednym z kluczowych aspektów produkcji muzycznej – strojeniu instrumentów strunowych, wykorzystując możliwości oferowane przez wtyczki VST. 

    Strojenie instrumentów strunowych jest podstawowym, lecz często niedocenianym elementem tworzenia muzyki. Dokładność strojenia wpływa nie tylko na jakość dźwięku, ale i na komfort pracy muzyków oraz ostateczny odbiór muzyczny przez słuchaczy. W kontekście technologicznym, proces ten stwarza unikalne wyzwania i możliwości. Z jednej strony, cyfrowe narzędzia do strojenia, takie jak tunery elektroniczne, są już szeroko stosowane, jednakże integracja tych narzędzi z zaawansowanymi funkcjami wtyczek VST otwiera nowe perspektywy dla precyzyjnego i efektywnego strojenia instrumentów. 

    Celem tej pracy jest zbadanie, w jaki sposób wtyczki VST mogą być wykorzystane do strojenia instrumentów strunowych, z uwzględnieniem ich unikalnych właściwości dźwiękowych i akustycznych. Praca ta ma na celu nie tylko opracowanie prototypu wtyczki VST przeznaczonej do strojenia, ale także analizę jej skuteczności, wydajności i użyteczności w praktyce muzycznej. 

    W ramach pracy dokonana zostanie przegląd istniejących technologii strojenia, analiza specyfikacji i wymagań dotyczących strojenia instrumentów strunowych, a także opracowanie i testowanie prototypu wtyczki VST. 

    Poprzez połączenie wiedzy technicznej z zastosowaniami praktycznymi, niniejsza praca inżynierska ma na celu zwiększenie zrozumienia roli technologii cyfrowych w muzyce oraz przyczynienie się do rozwoju narzędzi, które mogą znacząco wpłynąć na jakość i efektywność produkcji muzycznej.

\end{spacing} % Kończymy interlinię 1.5