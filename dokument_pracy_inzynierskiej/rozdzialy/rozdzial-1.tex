%-------------------------------------------------------------
%-------------------------------------------------------------
%-------------------------------------------------------------
%                          Rozdział 1
%-------------------------------------------------------------
%-------------------------------------------------------------
%-------------------------------------------------------------

\setstretch{1.5}
\setlength{\parindent}{1.25cm} % Ustaw wcięcie akapitu

\section{Podstawy teoretyczne}

\subsection{Historia i rozwój strojenia instrumentów}
\begin{spacing}{1.5} % Zaczynamy interlinię 1.5
    Lorem ipsum dolor sit amet, consectetur adipiscing elit. Praesent ultricies interdum eros, vitae condimentum libero porttitor semper. Nam mollis ante ut ipsum molestie, ut scelerisque diam consectetur. Class aptent taciti sociosqu ad litora torquent per conubia nostra, per inceptos himenaeos. Phasellus dui nunc, facilisis sit amet hendrerit quis, euismod sed felis. Aliquam dapibus ligula vel risus condimentum, vitae rutrum turpis aliquam. Mauris vitae purus in neque interdum auctor in ac sapien. Vivamus purus lectus, venenatis eget sollicitudin vel, fringilla a sem. Mauris vitae mollis mauris. Integer nec ex ac diam egestas sagittis. Quisque at neque ullamcorper metus sodales congue. Aenean nulla justo, gravida at suscipit at, vehicula eu ligula. 
\end{spacing} % Kończymy interlinię 1.5

\subsection{Fizyczne podstawy dźwięku instrumentów strunowych}
\begin{spacing}{1.5} % Zaczynamy interlinię 1.5
    Nam sapien leo, hendrerit sodales venenatis quis, dictum in sapien. Nunc eu sollicitudin metus. Maecenas vitae ipsum euismod, blandit nibh tincidunt, venenatis quam. Suspendisse a vehicula neque, eu fringilla massa. Phasellus sapien augue, ultricies et justo vitae, gravida venenatis leo. Aenean quis ex efficitur, scelerisque odio vitae, venenatis est. Aliquam condimentum vel velit elementum vulputate. Ut feugiat sit amet nisi vel gravida. Mauris faucibus neque eget commodo tristique. Vestibulum faucibus, tortor et facilisis finibus, tortor nunc tristique nisi, non scelerisque tellus turpis sed tellus. Pellentesque sed viverra tellus, eu vehicula lacus. Suspendisse et est et velit egestas rhoncus eget et nisl. Suspendisse fringilla euismod malesuada. 
\end{spacing} % Kończymy interlinię 1.5

\subsection{Technologie VST – wprowadzenie}
\begin{spacing}{1.5} % Zaczynamy interlinię 1.5
    Aenean eget condimentum elit, nec semper ante. Nunc ac hendrerit mi. Phasellus porttitor eros sit amet leo eleifend, quis condimentum mi iaculis. Phasellus aliquet eros quis enim finibus, in consectetur velit vehicula. Vestibulum placerat eleifend blandit. Mauris sed mattis magna. Praesent nunc ante, iaculis ac urna sit amet, tristique sollicitudin enim. 
\end{spacing} % Kończymy interlinię 1.5

\subsection{Charakterystyka instrumentów strunowych}
\begin{spacing}{1.5} % Zaczynamy interlinię 1.5
    Donec nisl urna, pulvinar vestibulum malesuada ut, pretium in tortor. Maecenas et tellus ut ipsum rhoncus consectetur ac ut enim. Nam commodo magna urna, ut ultricies nunc eleifend eu. Vestibulum in pharetra nisi. Integer non efficitur arcu, porta volutpat elit. Etiam nec ultrices libero, commodo rutrum ex. Vivamus ligula mauris, efficitur ac dolor nec, ultrices pellentesque nulla. Duis ultrices aliquam gravida. Nam tincidunt enim in metus vehicula, at laoreet est mollis. Mauris in justo in enim tincidunt egestas. Donec molestie varius dui, non imperdiet nunc cursus et. Aliquam erat volutpat. 
\end{spacing} % Kończymy interlinię 1.5

\subsection{Metody strojenia tradycyjnego}
\begin{spacing}{1.5} % Zaczynamy interlinię 1.5
    Nunc tincidunt cursus lacus, eu egestas libero imperdiet at. Quisque vitae dictum ex. Fusce convallis libero non dui volutpat, sit amet tristique risus pretium. Morbi ultricies scelerisque felis, a volutpat tortor volutpat ac. Vestibulum quis finibus orci. Suspendisse lobortis tristique mauris. Aenean ac euismod mauris. Vivamus erat justo, tempor quis dignissim ac, congue vitae velit. Orci varius natoque penatibus et magnis dis parturient montes, nascetur ridiculus mus. Vestibulum tristique faucibus mollis. Sed ultricies rhoncus leo sed imperdiet. Sed porta ipsum eu risus placerat, vitae viverra est lobortis. Duis tincidunt ligula vitae augue dignissim, id eleifend nulla ornare. Aenean ullamcorper hendrerit interdum. Integer sit amet vestibulum massa, ac eleifend purus. Lorem ipsum dolor sit amet, consectetur adipiscing elit. 
\end{spacing} % Kończymy interlinię 1.5

\subsection{Wprowadzenie do przetwarzania sygnału cyfrowego}
\begin{spacing}{1.5} % Zaczynamy interlinię 1.5
    Sed feugiat sollicitudin felis, at dictum metus. Etiam consectetur sem vitae leo iaculis euismod. Duis suscipit, mauris eget feugiat tincidunt, ipsum nisi molestie dolor, eu dapibus nibh ante id justo. Morbi id dapibus justo. Integer sit amet dui ante. Fusce orci tellus, convallis at fermentum in, vehicula non sapien. Donec scelerisque justo eget libero accumsan finibus. Aliquam sed luctus ante. Vestibulum purus arcu, posuere vitae auctor in, fermentum a ipsum. Pellentesque at nunc eget nisi feugiat sollicitudin. Mauris gravida nibh id nulla porta, quis viverra nibh condimentum. Donec ullamcorper aliquam tempor. Vestibulum ante ipsum primis in faucibus orci luctus et ultrices posuere cubilia curae; 
\end{spacing} % Kończymy interlinię 1.5

\subsection{Przegląd istniejących rozwiązań VST do strojenia}
\begin{spacing}{1.5} % Zaczynamy interlinię 1.5
    Sed enim neque, dapibus vitae blandit ut, tempus sit amet nibh. Vivamus imperdiet ullamcorper nibh id condimentum. Phasellus porttitor, enim eu pretium tincidunt, ligula lorem aliquam magna, at tincidunt purus libero eget ante. Quisque lectus eros, aliquet at pharetra vel, tincidunt quis enim. Ut quis libero vel nisl lacinia fermentum in et ipsum. Sed aliquet libero turpis, quis sollicitudin sem rutrum ut. Pellentesque pellentesque, arcu et blandit vestibulum, arcu arcu euismod libero, vitae ultrices nisl mi eu lacus. Aliquam erat volutpat. 
\end{spacing} % Kończymy interlinię 1.5

\subsection{Analiza potrzeb użytkowników i wymagania funkcjonalne}
\begin{spacing}{1.5} % Zaczynamy interlinię 1.5
    Sed rhoncus rutrum luctus. Nullam commodo non sem et consequat. Etiam tincidunt, tellus eu malesuada suscipit, tellus lacus egestas odio, maximus elementum turpis urna sed sem. Integer volutpat, urna quis lobortis tempus, ex est porttitor libero, vel fermentum felis purus vitae lacus. Morbi vel magna ac mi interdum facilisis ut vehicula orci. Vestibulum ligula sem, suscipit ut nunc vitae, aliquam laoreet justo. Sed at venenatis felis, non tempor elit. Cras et mauris consequat, efficitur risus a, gravida nulla. Proin dapibus blandit ullamcorper. Integer lacinia, diam in aliquet dignissim, diam nisl lobortis nunc, maximus gravida justo elit non libero. 
\end{spacing} % Kończymy interlinię 1.5
